\documentclass{article}
\usepackage[utf8]{inputenc}

% Modified this document to LaTeX:
% https://docs.google.com/document/d/13KyMaIwALko2bcrn195Mtr_WUd5ZyB0Ivm8-GJxYFn8/edit


\title{Bylaws of Sad Bee, Inc. (d/b/a Hive13)}
% Originally placed in LaTeX by uid0 (Ian Wilson) in 2013
% updated/verified on 20150120 by uid0.


\begin{document}
\maketitle
\section{Overview}
\subsection{Name}
The name of the corporation is Sad Bee, Inc..
\subsection{Purposes}
\begin{enumerate}
    \item Said corporation is organized exclusively for charitable, educational,
and scientific purposes within the meaning of \textbf{Section 501(c)(3)}\footnote{http://www.law.cornell.edu/uscode/text/26/501} of the Internal
Revenue Code, or the corresponding section of any future federal tax code.
    \item The corporation's mission is to promote technology through education and 
collaboration.
\end{enumerate}
\section{Membership}
\subsection{Membership Classes}
\begin{enumerate}
    \item Sad Bee, Inc. shall have levels of membership as defined by the
    Membership Addendum located at the following url: https://github.com/Hive13/bylaws-and-membership-addendum/raw/master/addendum.pdf which are
    subject to the approval of the membership by quorum vote.
\end{enumerate}
\subsection{Eligibility}
\begin{enumerate}
    \item In order to be a member, a person must apply via the membership
    application, support the purpose and scientific goals of the organization
    and must pay the monthly membership fee as determined the \textbf{Membership Addendum.}
    \item Any eligible person may be approved as a member, at any meeting with
    any board member or officer and upon payments of their first periodic dues,
    general approval of members present, and completion of the membership
    application form and waiver of liability.
    \item In order for a membership to be valid, the official membership
    application must be signed by an Officer or Director as defined by these
    bylaws.
    \item Members must be at least 18 years of age.
\end{enumerate}
\subsection{Rights and Responsibilities}
\begin{enumerate}
    \item All members shall have the right to:
    \begin{enumerate}
        \item vote on:
        \begin{enumerate}
            \item the election of directors and officers,
            \item any merger and its principal terms,
            \item any election to dissolve the corporation
            \item any issue put before the membership,
        \end{enumerate}
        \item voice their opinion and vote their preference or abstain from
        voting in the affairs of the corporation.
        \item In addition, all members shall have all rights afforded members
        under the law, and any other rights granted by resolution of the board
        of directors.
        \item any extra rights as listed under the \textbf{Membership Addendum.}
    \end{enumerate}
    \item All Members shall be responsible for:
    \begin{enumerate}
        \item timely payment of monthly dues as determined by the \textbf{Membership Addendum,}
        \item providing their current address, contact information, and
        preference for electronic receipt of communications,
        \item thoughtfully contributing to \textbf{Sad Bee, Inc.'s} direction and
    policies,
        \item continuing to support the purposes of the corporation,
        \item obeying any rules set forth by the board such as a noise curfew.
        \item At the time a member’s eligibility expires, he must forfeit his or
    her method of entry in addition to any other property owned by Sad Bee, Inc.
    to a member of the board of directors or an officer.
        \item any extra responsibilities as listed in the \textbf{Membership Addendum.}
    \end{enumerate}
\end{enumerate}

\subsection{Termination of Membership}
\begin{enumerate}
    \item A person ceases to be a member of the corporation
    \begin{enumerate}
    \item by delivering his or her resignation in writing the \textbf{Membership at Large,}
    \item on his or her death,
    \item on being expelled, or
    \item on having been a member not in good standing for 3 consecutive months, except by special arrangement at the discretion of the Board of Directors.
    \end{enumerate}
    \item Any member may resign by filing a written resignation with the \textbf{Board of Directors, Officers and Membership at large,} or by mailing or delivering it to the address of the corporation.
    \item Resignation shall not relieve a member of unpaid dues or other monies owed.
    \item Membership may also be terminated for any reason by resolution passed by more than \textbf{three quarters (3/4th)} of the voting members.
    \begin{enumerate}
        \item Notice of termination shall be given by any method reasonably calculated to provide actual notice to the member.
        \item The notice of special resolution for expulsion must be accompanied by a brief statement of the reasons for the proposed expulsion.
        \item The member shall be given an opportunity to be heard, either orally or in writing, before the effective date of the proposed termination.
        \item The hearing shall be held, or the written statement considered, by the members.
        \item The members shall then vote on whether the membership is to be terminated.
    \end{enumerate}
    \item The corporation reserves the right to limit membership based on the capacity of the space.
    \item Upon resignation or termination, members will have their membership rights and responsibilities revoked.
\end{enumerate}

\subsection{Suspension of Membership}
\begin{enumerate}
    \item Membership may be suspended for non-payment of dues by the \textbf{Board of Directors}.
    \item Any suspended member may restore their membership \textbf{90 days} after suspension upon payment of dues owed and payable through one month beyond the end of the suspension period.
    \item In order for the suspension to be lifted, the suspended member must go through the same vetting process as occurs on acceptance of a new member.
    \item A majority of the board can vote to suspend the membership of a member, at which time their access to the space will also be suspended.
\end{enumerate}
\subsection{Dues}
\begin{enumerate}
    \item The amount of the first membership dues must be determined by the \textbf{Board of Directors} and after that the monthly membership dues must be determined at the annual general meeting of the corporation.
    \item The \textbf{Board of Directors} shall determine the dues for members such that the corporation shall be financially sustained.
    \item All members are in good standing except a member who has failed to pay his or her current monthly membership fee, or any other subscription or debt due and owing by the member to the corporation, and the member is not in good standing so long as the debt remains unpaid.
\end{enumerate}
\section{Meetings}
\subsection{Regular Meetings}
\begin{enumerate}
    \item Regular meetings of the membership shall be held on a weekly basis.
    \item The meeting shall be held at the registered address or at a location determined by the \textbf{Board of Directors}.
    \item A different meeting place may be designated and agreed up by Votes of the Membership as defined in these Bylaws. Regular meetings of voting members shall be held \textbf{every Tuesday at 19:15} EST.
\end{enumerate}

\subsection{Annual Meetings}
\begin{enumerate}
    \item An annual meeting of members shall be held in the month of \textbf{July, beginning in 2009.} The President or their delegate shall fix the date, time,
    and location.
    \item \textbf{The Secretary} shall notify members as provided in the section of these
    bylaws entitled \textbf{Notice of Meetings}.
    \item The date and time can be changed by the procedures for a \textbf{Vote of the Membership}, as defined in these bylaws.
    \item Annual Meetings of the Membership exist in order to:
    \begin{enumerate}
        \item comply with legal requirements,
        \item elect the Board of Directors
        \item elect all officers,
        \item review and vote on the standing rules and policies of the
corporation,
        \item receive reports on the budget and activities of the corporation,
        approve the budget and determine the direction of corporation in the
        coming year.
        \item update the Bylaws, and
        \item any other issues that members have placed before the membership
        to be discussed at the annual meeting, pursuant to the proposal and
        voting rules stated in these bylaws for \textbf{Votes of the Membership}.
    \end{enumerate}
    \item The date, time and place of the annual meeting must be announced so as
    to give reasonable notice to members as described in the section of these
    bylaws entitled \textbf{Notice of Meetings}.
    \item Decisions will be by made by general consensus confirmed by vote,
    except for changes to bylaws which must be passed with \textbf{two-thirds majority}.
    \item Quorum for the annual meeting shall be \textbf{50\%}.
    \begin{enumerate}
        \item In absence of a quorum, no formal action shall be taken except to
        adjourn the meeting to a subsequent date.
    \end{enumerate}
\end{enumerate}

\subsection{Special Meetings}
\begin{enumerate}
    \item The \textbf{Board of Directors} or \textbf{five percent or more} of the members may call a special meeting of the members for any lawful purpose at any time.
    \item Notice must be provided of such meeting as provided in the section of these bylaws entitled Notice of Meetings.
\end{enumerate}
\subsection{Notice of Meetings}
\begin{enumerate}
    \item Notice of annual meetings of members, shall be written and shall be
    given at \textbf{least 10 but no more than 60 days} before the meeting date.
    \item Notice of regular meetings may be given personally, by email or any
    other means reasonably calculated to provide actual notice to all members.
    If email is used, notice shall be sent to the member at his or her email
    address shown in the corporation’s membership records.
    \item Special meetings require \textbf{72 hours} notice considered delivered only
    when all voting members are personally notified.
    \item For all meetings, the notice shall state the nature of the business to
    be transacted by the members.
    \item For a meeting where elections are held, the notice shall state the
    names of all persons who are nominees for office.
    \item The time and place of upcoming meetings shall be \textbf{conspicuously posted}
    at the registered office and electronically sent to all voting members.
    \item Announcements regarding changes to meeting date, time, or venue shall
    be made to the entire membership with a minimum of \textbf{48 hours} notice.
\end{enumerate}
\subsection{Remote Attendance}
\begin{enumerate}
    \item Members may participate in a meeting through use of conference
    telephone, electronic video screen communication, electronic chat, or other
    communications equipment so long as all of the following apply:
    \begin{enumerate}
        \item each member participating in the meeting can communicate with all
        of the other members;
        \item each member is provided with the means of participating in all
        matters under consideration, including the capacity to propose, or to
        interpose an objection to a specific action to be taken by the
        corporation;
        \item the Corporation verifies that
        \begin{enumerate}
            \item a person communicating by telephone, electronic video screen,
        or other communications equipment is a member with voting privileges,
        and
            \item all motions, votes, or other actions required to be made by a
        member were actually made by a member and not by someone who is not
        entitled to participate as a member.
        \end{enumerate}
    \end{enumerate}
    \item Votes by the Directors may not be made by proxy. Directors must attend
    the meeting of the Board of Directors, either in person or remotely through
    teleconference methods arranged with the Board before the meeting, in order
    to vote on matters placed before the Board of Directors.
\end{enumerate}
\section{Votes of the Membership}
All issues requiring a vote, except when otherwise specified in these bylaws, shall be decided by affirmative vote of \textbf{more than 50\% (one half)} of the quorum present.
\subsection{Quorum}
\begin{enumerate}
\item Quorum for a vote of the membership shall require attendance of at least \textbf{half (50\%)} of the existing membership on the day of the vote.
\item For the purposes of calculating the quorum, properly submitted proxy statements by members shall count as attendance as well as votes submitted via authenticated online polls as described in the section: \textbf{Remote Voting}.
\end{enumerate}
\subsection{Remote Voting}
A poll can be brought up online to handle collection of votes as long as the online voting system authenticates users as Hive members. Once the poll has been circulated via the mailing list there will be a period of no less than \textbf{seven (7) days} to vote on the topic. After the deadline the online poll is locked so no further votes are possible and the results are included with other voting methods described within the Bylaws to reach a quorum.




\section{Directors}
\subsection{Number}
\begin{enumerate}
  \item The Board of Directors shall serve without pay and consist of \textbf{five (5)}
  members.
  \item All directors must be a \textbf{Full Member} or equivalent of the corporation as
  defined in the Membership Addendum.
  \item Each director shall serve from the time of their election until their
  successor is elected and qualifies.
\end{enumerate}
\subsection{Compensation}
\begin{enumerate}
  \item A director must not be remunerated for being or acting as a director but:
  \begin{enumerate}
    \item A director must be reimbursed for all expenses necessarily and
    reasonably incurred by the director while engaged in the affairs of the corporation, and
    \item The corporation may provide insurance and indemnity as permitted by law.
  \end{enumerate}
\end{enumerate}
\subsection{Responsibilities}
\begin{enumerate}
  \item The duties of the Board shall include:
  \begin{enumerate}
    \item upholding and advancing the purposes of the corporation,
    \item being responsible for the legal, contractual, and financial affairs of
    the corporation,
    \item fulfilling all roles as required by Ohio law.
  \end{enumerate}
  \item Any policy affecting the organization at-large will, unless stated
  otherwise, be decided upon by the voting membership.
\end{enumerate}
\section{Elected Officers}
\subsection{Offices}
The officers of the Corporation shall consist of a President, a Secretary, a Treasurer, a 
Chief Technical Officer, a Chief Operations Officer, and such other officers as the board 
may from time to time deem advisable.  Any officer may be, but is not encouraged to be, a 
director of the Corporation.
\subsection{President}
\begin{enumerate}
  \item The president serves as a representative of the Corporation to the public and in all 
functions where a President may be called for law or any other outside requirements.
  \item The President shall organize, preside over, and set the agenda for all meetings of 
the membership and of the board of directors.
  \item The president is responsible for enforcing the rules of meeting procedure as 
detailed in these documents.
  \item The president shall facilitate communication between the membership at large, the 
officers, and the board of directors.
\end{enumerate}
\subsection{Secretary}
\begin{enumerate}
\item The Secretary shall supervise the keeping of a full and complete record of the 
proceedings of the Board of Directors and its committees.
\item The Secretary shall supervise the giving of such notices as may be proper or 
necessary.
\item The Secretary shall supervise the keeping of the minute books of this Corporation.
\item The Secretary shall be responsible for recording all minutes of all official meetings of 
the membership and the Board of Directors in the Corporation's permanent records.
\end{enumerate}
\subsection{Treasurer}
\begin{enumerate}
\item The treasurer is responsible for monitoring all financial assets of the Corporation.  
This includes, but is not limited to:
\begin{enumerate}
\item keeping record of the organization's budget,
\item the collection of membership dues from Members
\item the payment of rent and utilities from any space leased by the Corporation
\item the disbursement and reimbursement of funds authorized to be spend under the 
procedures detailed in these bylaws,
\item and preparing financial reports to the board.
\end{enumerate}
\item The Treasurer is responsible for making sure that the Corporation files its annual 
reporting and any other papers required to maintain legal nonprofit status by the law of 
Ohio or Federal Law
\end{enumerate}
\subsection{Chief Technical Officer}
\begin{enumerate}
\item The Chief Technical Office is responsible for ensuring the maintenance and 
consistency of the technological infrastructure as needed by the organization.  This 
includes, but is not limited to: the website and the internal network of the phyiscal space, and expanding it as necessary to adjust to the needs of the organization.
\end{enumerate}
\subsection{Chief Operations Officer}
\begin{enumerate}
\item The Chief Operations Officer is responsible for managing the safety, security, and 
tidiness of the physical space.
\item The Chief Operations Officer is responsible for providing logistical support to events 
such as meetings, classes, workshops, and parties.
\end{enumerate}
\subsection{Officer Duties}
\begin{enumerate}
\item In their areas of responsibility, Officers are expected to build consensus and work 
toward the goals of the Corporation and its Members.
\item Officers may enlist the help of other members and non-members in meeting their 
responsibilities.
\item At the end of their term, Officers are expected to facilitate with the transition 
process by training the newly elected Officers and helping them become familiar with any 
day-to-day duties that are not explicitly described in the bylaws or otherwise.
\end{enumerate}
\subsection{Eligibility}
In order to be eligible to be Nominated as an officer or a director at large, a person must 
be a Full Member (or equivalent) as defined earlier in the Membership Addendum for three 
consecutive months.
\subsection{Nomination}
\begin{enumerate}
\item Any qualified member has the right to nominate an eligible person for office.
\item Any qualified member has the right to nominate themselves.
\item Only the nominated candidate can un-nominate themselves.
\item All nominations for director and officer positions are due one week before the annual 
meeting.
\item If only one person is timely nominated to run for an office and accepts such 
nomination, they shall run unopposed.
\item If no person is timely nominated to run for an office, nominations for that position 
may be made at the annual meeting, in person, before the vote. If nobody is nominated in 
this way, the incumbent may choose to continue in the position or choose to appoint a 
willing successor.
\item In the event that there is an unfilled officer position at the end of an election, the 
board may appoint an interim officer that shall serve until a replacement officer is properly 
elected. These interim officers shall serve no longer than 30 days. If the position is still 
vacant after the interim, the board must re-appoint a member to the unfilled position.
\end{enumerate}
\subsection{Elections}
\begin{enumerate}
\item Elections for the board of directors shall take place at the annual meeting or another 
meeting with quorum.
\item All eligible positions shall be elected at the same time, by the process determined in 
these bylaws for the Votes of the Members, with one exception:  Full members may cast up 
to three votes for the directors at large;  these votes do not have to be for separate 
candidates.
\item If there is more than one candidate for the position, the candidate which obtains the 
highest number votes from voting members present shall be elected;  in the case of the 
directors, the candidates receiving the 5 highest vote tallies shall be elected.
\end{enumerate}
\subsection{Resignations and Terminations}
\begin{enumerate}
\item Any Officer or Director may resign at any time by written notice delivered to the 
Board of Directors, Officers, and Membership at Large of the corporation.
\item A resignation is effective when the notice is delivered unless the notice specifies a 
future date.
\item Any Officer or Director may be terminated in their role by resolution passed by Vote 
of the Membership.
\item Nominations for people to run to replace the officer who base resigned shall open 
when the officeholder tenders resignation, and remain open for one week.
\item Members shall elect the replacement among the candidates who have been 
nominated and accepted their nomination using the Votes of the Membership procedures 
in these bylaws.
\item The replacement's term shall last until the next Annual Meeting.
\end{enumerate}
\subsection{Term}
The term for each elected office expires at the next Annual Meeting.
\section{Committees}
\begin{enumerate}
\item Committees may be formed by the membership at large at any time, provided:
\begin{enumerate}
\item The committee does not understate any action that can jeopardize the legal status 
of the corporation.
\item The committee does not understate any action that can jeopardize the safety or 
security of the Membership at Large or The Corporation.
\end{enumerate}
\item A Chairperson for the committee may be chose from among the committee 
members.
\begin{enumerate}
\item Officers and Board Members server as committee chairpersons.
\end{enumerate}
\item Committees will be expected to provide updates on their activities to the 
Membership at large via reports at the weekly meetings, updates via the mailing list, or any 
other effective method that the Committee deems appropriate.
\item Meetings of committees may be held without notice at such time and place as shall 
from time to time be determined by the committees.
\item The Committees of the corporation shall keep regular minutes and report these 
minutes to the Membership at Large, the Officers, and the Board of Directors.
\end{enumerate}
\section{Books, Records, and Reports}
\subsection{Annual Report}
The Corporation shall send an annual report to the Members of the Corporation not later 
than 6 months after the close of each fiscal year of the Corporation. Such report shall 
include a balance sheet as of the close of the fiscal year of the Corporation and a revenue 
and disbursement statement for the year ending on such closing date.  Such financial 
statements shall be prepared from and in accordance with the books of the Corporation, 
and in conformity with generally accepted principles applied on a consistent basis.
\subsection{Permanent Records}
The corporate shall keep current and correct records of the accounts, minutes of the 
meetings and proceedings, and membership of the corporation.  Such records shall be kept 
at the registered office or the principal place of business of the corporation.  Any such 
records shall be in written form or in a form capable of being converted into written form.
\subsection{Inspection of Corporate Records}
Any person who is a Member of the corporation shall have the right at any reasonable time, 
and on written demand stating the purpose thereof, to examine and make copies from the 
relevant books and records of accounts, minutes, and records of the Corporation.  Upon 
the written request of any Member, the Corporation shall mail such a member a copy of the 
most recent balance sheet and revenue disbursement statement.
\section{Fiscal Year}
The fiscal year of the corporation shall be the period selected by the Board of Directors as 
the tax year of the corporation for federal income tax purposes.
\section{Corporate Seal}
The Board of Directors may adopt, use, and modify a corporate seal.  Failure to affix the 
seal to corporate documents shall not affect the validity of such documents.
\section{Indemnification}
Any officer, director, or employee of the corporation shall be indemnified to the full extent 
allowed by laws of the State of Ohio.
\section{Amendments}
\begin{enumerate}
\item These bylaws shall be amended by a \textbf{two-thirds} Vote of the Membership present at 
any Annual or Special member meeting provided a quorum is present and provided a copy 
of the proposed amendment(s) are provided to teach member with the agenda for the 
meeting, using the procedures stated in the Votes of the Membership section of these 
bylaws.
\item Proposed amendments to these Bylaws shall be submitted in writing to the 
Membership at least one week in advance of the meeting at which they will be considered 
for adoption.
\end{enumerate}
\section{Conflict of Interest}
Any member of the board who has a financial, personal, or official interest in, or conflict 
(or appearance of a conflict) with any matter pending before the Board, of such nature that it 
prevents or may prevent that member from acting on the matter in an impartial manner, 
will offer to the Board to voluntarily excuse himself or herself and will vacate his or her seat 
and refrain from discussion and voting on said item.
\section{Certification}
This shall certify that the attached is a true and correct copy of the bylaws of this 
corporation, and that such bylaws were duly adopted by the Incorporator and approved by 
the Board of Directors of this corporation.

Dated:

Signed:

Printed Name:

\end{document}
