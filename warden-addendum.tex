\documentclass[11pt, oneside]{article}   	% use "amsart" instead of "article" for AMSLaTeX format
\usepackage{geometry}                		% See geometry.pdf to learn the layout options. There are lots.
\geometry{letterpaper}                   		% ... or a4paper or a5paper or ... 
%\geometry{landscape}                		% Activate for for rotated page geometry
%\usepackage[parfill]{parskip}    		% Activate to begin paragraphs with an empty line rather than an indent
\usepackage{graphicx}				% Use pdf, png, jpg, or eps� with pdflatex; use eps in DVI mode
								% TeX will automatically convert eps --> pdf in pdflatex		
\usepackage{amssymb}

\title{Warden Addendum}
\author{Jon Neal (jon@hive13.org)}
\author{Greg Arnold (garnold@hive13.org)}
\date{}							% Activate to display a given date or no date

\begin{document}
\maketitle
%\section{}
%\subsection{}
\section{Summary}
Area wardens are active members who want to become experts of a ward and be looked up
to as the representation of the Hive mission.  Area wardens are able to effectively offer
advice, teach users how to properly and safely use their machines, maintain their space,
and directly request replacements or improvements to items in their space. The increased
responsibility of these members comes with the benefit of a reduced membership fee and
direct access to funds for improvement of the space.  Area wardens also include other
aspects of the Hive mission, including education.
\section{Organization}
The COO has control of managing area wardens and their financial capabilities.  The COO
is responsible for ensuring that area wardens are respecting their ward of responsibility. The
COO is responsible for appointing and removing area wardens. Additionally, the COO is
responsible for purchase and approval of funding for consumables/replacements from the
warden budget allocated by the membership.  Short monthly progress meetings will be held
with the wardens.
\section{Budget}
A total of \$250 is allocated each month, in addition to the compensation given to the
wardens. Area wardens have direct access to a spreadsheet for requesting
parts/consumables for their area out of this budget. The COO makes the decision whether
these will be purchased from the warden budget or should be voted on by the Hive
membership at a regular meeting. Items are purchased by the COO, and the monthly summary
is presented to the Board at each board meeting for approval and reimbursement to the COO.
\section{Compensation}
Area wardens can choose to be compensated 50\% of dues for full members and 100\% of
dues for student members (one dollar for accounting purposes).  If one member is
appointed to multiple warden roles, no additional compensation is provided.  If
the COO serves in a warden role, he is not entitled to this compensation.
\section{Responsibilities}
Area wardens are to:
\begin{itemize}
\item Post a picture with a name and quick summary of who they are as well as at least two reliable forms of contact (i.e. phone number, irc handle, email) to be contacted at in the case of questions or machine failure.
\item Ensure that their space is kept clean (proactively and reactively).
\item Keep machines reasonably maintained.
\item Maintain Wiki entries pertaining to any and all items in each area.
\item Provide documentation and certification classes on major tools in each area.
\item Consumables/parts are replenished through the warden budget.
\item Hold public open shop hours at least once a month at which the warden is present for 3-4 hours if members want to learn or receive help on a particular machine in their ward.  The lounge/meeting area warden should organize a social activity such as a movie night instead.
\item Provide feedback on how to improve the warden system.
\end{itemize}
The educational warden serves as the central hub of the Hive's educational activites, including:
\begin{itemize}
\item Scheduling classes with the input of the person hosting the class
\item Announcing the class through various outlets such as social media, flyers, announcements, and any other method
\item Handling payments and registration, if the class requires either or both.
\end{itemize}
\section{Area Wardens}
\begin{itemize}
\item CNC Machining
\item Woodworking
\item Metalworking/Small Tools/Dirty Room/Welding
\item Digital Fabrication Lounge (CAD/3D printing/Laser cutters)
\item Lounge/Meeting Area
\item Kitchen Area
\item Electronics Area
\item Educational warden
\end{itemize}
\end{document}  
